\section{What is a feedforward neural network?}
A Feedforward Neural Network (FNN) is a type of Artificial Neural Network (ANN) where the information flows in only one direction, from the input layer to the output layer, without any feedback connections. In a feedforward neural network, the input data is passed through multiple hidden layers of neurons, each with its own set of weights and biases, to produce the final output.

The architecture of a feedforward neural network consists of three types of layers:

Input layer: This layer receives the input data and passes it on to the hidden layers for processing. The number of neurons in the input layer depends on the dimensionality of the input data.

Hidden layers: These layers process the input data by applying non-linear transformations to the weighted sum of the inputs from the previous layer. The number of hidden layers and the number of neurons in each hidden layer depend on the complexity of the problem and the size of the input data.

Output layer: This layer produces the final output of the network by applying a final non-linear transformation to the weighted sum of the inputs from the last hidden layer. The number of neurons in the output layer depends on the dimensionality of the output data.

Feedforward neural networks are trained using the backpropagation algorithm, where the network is first initialized with random weights and biases, and the backpropagation algorithm is used to iteratively adjust the weights and biases to minimize the difference between the predicted output and the true output.

Feedforward neural networks have been successfully applied to many real-world problems, such as image classification, speech recognition, natural language processing, and financial forecasting. However, they may not be suitable for problems that require feedback connections, such as in the case of sequential data and time-series analysis.

