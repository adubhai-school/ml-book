\section{What is a hyperparameter tuning framework?}
A hyperparameter tuning framework is a software tool or library that provides an efficient and user-friendly way to optimize the hyperparameters of a machine learning model. A hyperparameter tuning framework typically provides a set of algorithms and strategies for hyperparameter search, a search space for hyperparameters, and a method for evaluating the performance of the models. Some popular hyperparameter tuning frameworks include:

1. Grid Search: A simple but exhaustive search algorithm that explores all combinations of hyperparameters in a predefined search space.

2. Random Search: A more efficient search algorithm that randomly samples hyperparameters from the search space and evaluates the performance of the resulting models.

3. Bayesian Optimization: A probabilistic approach to hyperparameter search that models the performance of the model as a function of the hyperparameters and uses a Bayesian model to guide the search.

4. Genetic Algorithms: An evolutionary approach to hyperparameter search that uses a population-based algorithm to explore the search space.

5. Hyperopt: A Python library that provides an efficient implementation of Bayesian Optimization and other hyperparameter search algorithms.

6. Ray Tune: A distributed hyperparameter tuning framework that supports a variety of hyperparameter search algorithms and integrates with popular machine learning libraries.

7. Optuna: A hyperparameter optimization framework that supports a wide range of search algorithms and integrates with popular machine learning libraries.

Hyperparameter tuning frameworks can help automate the process of hyperparameter search and improve the performance of machine learning models by finding the optimal hyperparameters for a given task.

