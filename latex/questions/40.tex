\section{What is batch normalization?}
Batch normalization is a technique used in machine learning and deep learning to improve the stability and performance of neural networks. It works by normalizing the input data of each layer to have zero mean and unit variance across the mini-batch of samples in the training set.

During the training phase, batch normalization calculates the mean and standard deviation of the input data for each layer and applies a scaling and shifting transformation to normalize the data. This transformation is learned during training and applied during inference to normalize the input data to each layer.

The main benefits of batch normalization are:

1. Improved stability and convergence: By normalizing the input data to each layer, batch normalization reduces the effects of covariate shift and internal covariate shift, which can cause the optimization process to become unstable or slow down.

2. Regularization: Batch normalization acts as a form of regularization by adding noise to the input data, which helps prevent overfitting and improve the generalization of the model.

3. Increased learning rates: Batch normalization enables the use of higher learning rates during training, which can lead to faster convergence and better performance.

Batch normalization is widely used in deep learning applications, particularly in computer vision and natural language processing, and has been shown to improve the accuracy and speed of training deep neural networks.

