\section{What is a hyperparameter scheduler?}
A hyperparameter scheduler is a technique used in machine learning to dynamically adjust the values of hyperparameters during the training process. Hyperparameters are parameters that are set before the training process begins, and they are not learned during training like model parameters.

A hyperparameter scheduler can be used to automatically adjust the values of hyperparameters based on the progress of the training process. For example, the learning rate hyperparameter can be adjusted based on the current loss or accuracy of the model, or based on the number of iterations or epochs that have passed.

Hyperparameter scheduling can help improve the performance of a model by finding the optimal values of hyperparameters during training. It can also help prevent overfitting by adjusting hyperparameters as the training progresses to ensure that the model is not fitting the training data too closely.

There are various types of hyperparameter schedulers, including step schedules, polynomial schedules, and exponential schedules. Step schedules adjust the hyperparameters at fixed intervals, while polynomial and exponential schedules adjust the hyperparameters based on a function of the current iteration or epoch. The choice of scheduler depends on the problem at hand and the characteristics of the training process.

