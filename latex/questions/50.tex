\section{What is a probability distribution?}
A probability distribution is a mathematical function that describes the likelihood of each possible outcome of a random variable in a probabilistic system. In other words, it is a way of representing the probability of each possible value of a random variable.

Probability distributions can be divided into two main categories: discrete distributions and continuous distributions.

Discrete probability distributions describe the probabilities of discrete events or outcomes, such as the number of heads in a coin toss or the number of people in a room. Examples of discrete distributions include the binomial distribution, the Poisson distribution, and the Bernoulli distribution.

Continuous probability distributions describe the probabilities of continuous events or outcomes, such as the height or weight of a person or the temperature of a room. Examples of continuous distributions include the normal distribution, the exponential distribution, and the uniform distribution.

Probability distributions are characterized by various parameters, such as mean, variance, and standard deviation, which describe the central tendency and the spread of the distribution. Probability distributions can be used in many fields, such as statistics, physics, engineering, and finance, to model and analyze various phenomena and make predictions about future events.

