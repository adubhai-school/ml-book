\section{What is model fine-tuning?}
Model fine-tuning is a technique used in machine learning to improve the performance of a pre-trained model by adapting it to a new task or dataset. The basic idea is to take a pre-trained model that has already learned to recognize a set of features, and then modify it to better suit the new task or dataset.

In model fine-tuning, the pre-trained model is typically used as a starting point, and then the weights and parameters of the model are adjusted during training to better fit the new data. This can involve changing the architecture of the model, modifying the hyperparameters, or adjusting the weights of specific layers.

The main advantage of model fine-tuning is that it allows us to take advantage of the knowledge learned by a pre-trained model, which can be especially useful in cases where the new task or dataset is related to the original task or dataset used for pre-training. For example, a pre-trained model that has learned to recognize objects in natural images can be fine-tuned to recognize specific types of objects in medical images.

Model fine-tuning is widely used in many applications, including computer vision, natural language processing, and speech recognition. Some popular pre-trained models that are often fine-tuned include the VGG, ResNet, and Inception convolutional neural networks for image recognition, and the BERT and GPT-2 models for natural language processing.

