\section{What is a hyperparameter?}
In machine learning, a hyperparameter is a parameter whose value is set before the start of the learning process. Unlike model parameters, which are learned during the training process, hyperparameters must be set by the user or a tuning algorithm. Examples of hyperparameters include the learning rate, regularization parameter, number of hidden layers, and number of neurons per layer in a neural network.

The choice of hyperparameters can have a significant impact on the performance of a machine learning model. The process of selecting optimal hyperparameters is known as hyperparameter tuning or hyperparameter optimization. This can be done through a variety of methods, including manual tuning, grid search, random search, and Bayesian optimization.

Hyperparameter tuning is an important part of the machine learning workflow, as it can significantly improve the accuracy and generalization of a model.

