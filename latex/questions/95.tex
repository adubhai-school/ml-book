\section{What is the difference between a Monte Carlo tree search and a deep Q-network?}
Monte Carlo tree search (MCTS) and deep Q-network (DQN) are two different approaches to reinforcement learning that have been successful in solving different types of problems.

MCTS is a model-based approach that uses a search tree to simulate possible future states of the environment and evaluate the expected outcomes of different actions. It builds a tree by repeatedly simulating trajectories from the current state and selecting actions based on the expected reward and uncertainty of each node. MCTS is particularly effective in problems with large state spaces and complex action spaces, such as board games and robotics.

DQN, on the other hand, is a model-free approach that learns a Q-function to estimate the value of taking an action in a given state. It uses a deep neural network to approximate the Q-function, which allows it to handle high-dimensional state spaces. DQN is particularly effective in problems where the state space is continuous or where the dynamics of the environment are unknown, such as video games and robotics.

One of the main differences between MCTS and DQN is the type of exploration they use. MCTS explores the environment by simulating possible futures, while DQN explores the environment by selecting actions that are expected to maximize the Q-value. Another difference is the way they handle uncertainty. MCTS explicitly models the uncertainty of the environment, while DQN uses a replay buffer to store past experiences and learn from them.

Overall, MCTS and DQN are complementary approaches that can be used together to solve complex reinforcement learning problems. MCTS can provide a powerful planning mechanism to explore the environment and make long-term decisions, while DQN can provide a fast and efficient way to learn a policy in high-dimensional state spaces.

