\section{What is GloVe?}
GloVe, short for Global Vectors, is an unsupervised learning algorithm for generating word embeddings, introduced by Stanford University researchers in 2014. GloVe uses a co-occurrence matrix to capture the distributional patterns of words in a corpus and then factorizes the matrix to learn the word embeddings.

The co-occurrence matrix is constructed by counting how often pairs of words co-occur in a large corpus of text. The entries in the matrix represent the number of times two words appear together in the same context. The matrix is then factorized using a technique called singular value decomposition (SVD) to learn a low-dimensional representation of the co-occurrence matrix that captures the semantic and syntactic relationships between words.

The resulting word embeddings are dense vectors that capture the distributional patterns of words in the corpus. Words that appear in similar contexts are expected to have similar vector representations, allowing the embeddings to capture the underlying relationships between words.

GloVe has been shown to outperform other word embedding algorithms in a number of natural language processing tasks, such as word analogy and word similarity tasks. It has become a popular choice for generating word embeddings and is widely used in both academia and industry.

