\section{What is the difference between supervised and unsupervised learning?}
The main difference between supervised and unsupervised learning lies in the type of input data that is used to train the Machine Learning model.

\subsection{- Supervised Learning:}
Supervised Learning is a type of Machine Learning where the algorithm learns from labeled data. This means that the data provided to the algorithm during training has the correct output values already assigned. The algorithm learns to make predictions or classify new data based on these labeled examples.

Supervised learning is used for tasks where the goal is to predict an output variable based on input variables. For example, predicting the price of a house based on its square footage, location, and other features. Some common supervised learning algorithms are Linear Regression, Logistic Regression, Decision Trees, Random Forest, Support Vector Machines, and Neural Networks.

\subsection{- Unsupervised Learning:}
Unsupervised Learning is a type of Machine Learning where the algorithm learns from unlabeled data. This means that the data provided to the algorithm during training has no pre-determined output values. The algorithm learns to find patterns, structures, and relationships in the data on its own.

Unsupervised learning is used for tasks where the goal is to discover hidden patterns, group similar data points together, or reduce the dimensionality of the data. Some common unsupervised learning algorithms are K-Means Clustering, Hierarchical Clustering, Principal Component Analysis (PCA), and Association Rule Mining.

In summary, supervised learning requires labeled data to train the model, while unsupervised learning works with unlabeled data to discover patterns and structures on its own.

