\section{What is a loss function?}
A loss function, also known as a cost function, is a mathematical function that measures the difference between the predicted output and the actual output of a machine learning algorithm. It quantifies how well the algorithm is performing on the training data, and it is used to optimize the algorithm's parameters during the training phase.

The loss function typically takes the predicted output of the algorithm and the actual output of the training data as input and returns a scalar value that represents the error or difference between them. The goal of the machine learning algorithm is to minimize the value of the loss function by adjusting its parameters, such as the weights and biases of a neural network.

The choice of the loss function depends on the type of machine learning problem, such as classification or regression, and the performance metric of interest, such as accuracy, precision, recall, or F1 score. Common examples of loss functions include mean squared error (MSE) for regression problems, binary cross-entropy for binary classification problems, and categorical cross-entropy for multi-class classification problems.

The loss function plays a critical role in the training of machine learning algorithms because it guides the optimization process and determines the quality of the final model. A well-designed loss function can help the algorithm converge faster and achieve better performance on the validation and test data.

