\section{What is a restricted Boltzmann machine?}
A restricted Boltzmann machine (RBM) is a type of neural network that models the probability distribution of a set of binary or real-valued data. It is a variant of the Boltzmann machine that has a restricted architecture, meaning that the nodes are divided into two layers: visible and hidden, and there are no connections between nodes within the same layer.

The RBM consists of a set of visible nodes and hidden nodes, connected by weighted edges. The nodes can take on binary or real-valued values, and the weights represent the strength of the interaction between the nodes.

The RBM is trained using a learning algorithm called Contrastive Divergence, which involves updating the weights based on the difference between the observed data and the reconstructed data generated by the model. The learning algorithm uses stochastic gradient descent to find the weights that maximize the log-likelihood of the observed data.

The RBM can be used for a variety of tasks, such as feature extraction, data compression, and generation of new samples from the learned distribution. It has many desirable properties, such as the ability to model complex dependencies between variables, the ability to handle missing or incomplete data, and the ability to learn useful representations of the data.

However, the RBM also has some limitations, such as the sensitivity to the choice of hyperparameters, the difficulty of training the model for large datasets, and the need for careful tuning of the learning algorithm. These limitations can be addressed by using more advanced variants of the RBM, such as the Deep Belief Network or the Autoencoder.

