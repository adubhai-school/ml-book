\section{What is a pooling layer?}
A Pooling Layer is a type of layer in Convolutional Neural Networks (CNNs) that reduces the spatial dimensionality of the feature maps produced by the convolutional layer. The goal of pooling is to extract the most important features from the feature maps while reducing the computational complexity of the network and avoiding overfitting.

The most common type of pooling operation is Max Pooling, where a filter or kernel of fixed size is moved over the feature map, and the maximum value within each subregion is taken as the output. Max Pooling reduces the spatial size of the feature maps by downsampling them, while retaining the most important features and preserving the location of the features in the input.

Another type of pooling operation is Average Pooling, where the average value of each subregion is taken as the output. Average Pooling is less commonly used in CNNs but can be useful in certain applications where Max Pooling may discard too much information.

Pooling layers are typically inserted after the convolutional layers in CNNs to reduce the spatial dimensionality of the feature maps and increase the translational invariance of the network to small changes in the input. The output of the pooling layer is then fed into a fully connected layer or another convolutional layer for further processing.

Pooling layers can also help to reduce the overfitting of the network by reducing the number of parameters in the network and providing a form of regularization. However, excessive pooling can also result in a loss of information and reduced accuracy, so the size and number of pooling layers should be carefully chosen based on the specific task and input data.

