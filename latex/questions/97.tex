\section{What is a genetic algorithm?}
A genetic algorithm is a type of optimization algorithm inspired by the process of natural selection in biology. It is used to find the optimal solution to a problem by simulating the evolution of a population of candidate solutions over many generations.

The genetic algorithm works by generating an initial population of candidate solutions, typically randomly. Each candidate solution is represented as a string of parameters, often called a chromosome. These parameters are subject to genetic operators such as crossover and mutation, which combine and modify the chromosomes to create new candidate solutions. The fitness of each candidate solution is evaluated based on how well it solves the problem, and the fittest solutions are selected for reproduction to create the next generation of candidates.

The process of selection, crossover, and mutation is repeated for many generations until an optimal solution is found or a stopping criterion is reached. Genetic algorithms can be used to optimize a wide range of problems, including engineering design, scheduling, and financial portfolio management.

In contrast to supervised and reinforcement learning, genetic algorithms do not require labeled data or a reward signal. Instead, they optimize a function directly based on a fitness function.

