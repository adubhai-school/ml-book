\section{What is a hyperparameter tuning tool?}
A hyperparameter tuning tool is a software tool or framework that automates the process of hyperparameter tuning for machine learning models. The tool typically provides a set of algorithms or strategies for searching the hyperparameter space, such as grid search, random search, or Bayesian optimization. It also allows the user to define the search space of hyperparameters and specify the objective function to optimize.

Some popular hyperparameter tuning tools include:

1. Hyperopt: a Python library for hyperparameter optimization that uses Bayesian optimization algorithms.

2. Optuna: a Python library for hyperparameter optimization that uses a combination of various optimization algorithms, including TPE and CMA-ES.

3. Ray Tune: a Python library for distributed hyperparameter tuning that supports a variety of search algorithms and integrates with popular machine learning frameworks such as PyTorch and TensorFlow.

4. GridSearchCV: a function in the Scikit-learn library for performing grid search over a specified range of hyperparameters.

5. RandomizedSearchCV: a function in the Scikit-learn library for performing random search over a specified range of hyperparameters.

These tools can save a lot of time and effort in hyperparameter tuning, especially for complex models with many hyperparameters or for large datasets where manual tuning may not be feasible.

