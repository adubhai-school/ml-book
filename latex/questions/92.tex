\section{What is the exploration-exploitation tradeoff?}
The exploration-exploitation tradeoff is a fundamental concept in reinforcement learning and decision-making that refers to the balance between exploring new options (exploration) and exploiting the current best option (exploitation) to maximize the expected cumulative reward over time.

In reinforcement learning, exploration refers to the process of taking actions that are not based solely on the current estimate of the optimal policy, but rather are aimed at gaining more information about the environment and the rewards associated with different actions. By exploring new options, the agent can discover better ways to perform the task and potentially achieve higher long-term rewards.

Exploitation, on the other hand, refers to the process of taking actions that are based on the current estimate of the optimal policy, with the aim of maximizing the immediate reward. By exploiting the current best option, the agent can achieve immediate rewards and potentially avoid making costly mistakes.

The exploration-exploitation tradeoff arises because in order to achieve high long-term rewards, the agent must balance the need to explore new options and the need to exploit the current best option. If the agent focuses too much on exploration, it may not achieve high immediate rewards and may take too long to discover the optimal policy. Conversely, if the agent focuses too much on exploitation, it may get stuck in a suboptimal policy and miss out on potentially better options.

The optimal balance between exploration and exploitation depends on the specific problem at hand and can be influenced by factors such as the complexity of the environment and the level of uncertainty in the rewards associated with different actions. Various techniques, such as epsilon-greedy, Thompson sampling, and UCB, have been developed to address the exploration-exploitation tradeoff and achieve better long-term performance in reinforcement learning.

