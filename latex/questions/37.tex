\section{What is the difference between a tensor and a matrix?}
In mathematics and computer science, a tensor is a generalization of a matrix to multiple dimensions. A matrix is a two-dimensional array of numbers or symbols that can be used to represent linear transformations or systems of linear equations.

The main differences between a tensor and a matrix are:

1. Dimensionality: A tensor can have any number of dimensions, whereas a matrix is limited to two dimensions.

2. Size: A tensor can have varying sizes along each dimension, whereas a matrix has fixed sizes along both dimensions.

3. Representation: A tensor can represent more complex data structures, such as images, videos, and time series, whereas a matrix is primarily used to represent linear transformations.

4. Operations: Tensors support a wider range of operations than matrices, such as element-wise operations, tensor contractions, and tensor products.

In machine learning and deep learning, tensors are the primary data structure used to represent and manipulate data, particularly in multi-dimensional arrays. Tensors are used to represent various types of data, such as images, videos, and time series, and are used as inputs and outputs of neural network models. Matrices are also used in machine learning and deep learning, particularly in linear algebra operations such as matrix multiplication, but they are not as common or versatile as tensors.

