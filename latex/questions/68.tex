\section{What is a denoising autoencoder?}
A denoising autoencoder is a type of neural network architecture that is used for unsupervised learning of features in noisy data. The denoising autoencoder learns to remove noise from a corrupted input data point, and reconstruct a clean output data point that is similar to the original input.

The denoising autoencoder consists of two parts: an encoder network and a decoder network. The encoder network takes an input data point that has been corrupted with noise and maps it to a low-dimensional latent space. The decoder network takes a point in the latent space and maps it back to the original input space, generating a synthetic data point that is similar to the original input but with the noise removed.

During training, the denoising autoencoder is trained to minimize the reconstruction error between the clean output data point and the original input data point. By minimizing this error, the denoising autoencoder learns to remove noise from the input data point and generate a clean output data point.

The denoising autoencoder can be useful for a wide range of applications, including image and signal processing, speech recognition, and natural language processing. However, training denoising autoencoders can be challenging, particularly when dealing with complex, high-dimensional data. To overcome this challenge, various regularization techniques, such as dropout and weight decay, are often used to prevent overfitting and improve generalization.

