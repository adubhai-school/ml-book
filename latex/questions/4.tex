\section{Explain the bias-variance trade-off.}
The bias-variance trade-off is a fundamental concept in Machine Learning that describes the balance between the complexity of a model and its ability to generalize to new data.

Bias refers to the error that occurs when a model makes assumptions about the data that are too simple or too restrictive. For example, if we fit a linear regression model to data that has a non-linear relationship, the model will have high bias and will not be able to capture the complexity of the underlying relationship.

Variance refers to the error that occurs when a model is too complex and captures the noise in the data instead of the underlying pattern. For example, if we fit a high-degree polynomial regression model to data with a simple linear relationship, the model will have high variance and will not generalize well to new data.

The bias-variance trade-off states that as we increase the complexity of a model, the bias decreases but the variance increases. Conversely, as we decrease the complexity of a model, the bias increases but the variance decreases. The optimal model has a balance between bias and variance that minimizes the total error, which is called the generalization error.

To find the optimal balance between bias and variance, we can use techniques like cross-validation to evaluate the performance of different models and choose the one with the lowest generalization error.

In summary, the bias-variance trade-off is a key concept in Machine Learning that helps us find the optimal balance between model complexity and generalization performance. By understanding this trade-off, we can choose the appropriate model for a given task and avoid overfitting or underfitting the data.

