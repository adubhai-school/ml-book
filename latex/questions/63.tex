\section{What is a transformer?}
A transformer is a type of neural network architecture that was introduced in 2017 in a paper called "Attention Is All You Need". It was primarily designed for natural language processing tasks such as language translation, but has since been applied to a wide range of tasks including image recognition and speech processing.

The transformer architecture is based on the concept of self-attention, which allows the network to selectively attend to different parts of the input sequence to extract relevant features. This is done by computing a weighted sum of the input at each position, where the weights are determined by a learned attention mechanism.

The transformer consists of an encoder and a decoder, each composed of multiple layers of self-attention and feedforward neural networks. The encoder takes the input sequence and transforms it into a sequence of hidden representations, while the decoder generates the output sequence by attending to the encoder hidden states and predicting the next token in the output sequence.

One of the key advantages of the transformer architecture is that it can capture long-range dependencies between different parts of the input sequence more efficiently than previous neural network architectures such as recurrent neural networks (RNNs) or convolutional neural networks (CNNs). This makes it well-suited for tasks involving long input sequences, such as language translation or speech recognition.

The transformer has become one of the most popular neural network architectures in natural language processing, and has been used in many state-of-the-art models for tasks such as language translation, text classification, and text generation.

