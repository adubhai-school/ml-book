\section{What is the F1 score?}
The F1 score is a commonly used performance metric in Machine Learning that combines precision and recall into a single score. It is used to evaluate the accuracy and completeness of the predictions made by a binary classification model.

The F1 score is the harmonic mean of precision and recall and is calculated as:

`F1 score = 2 * (precision * recall) / (precision + recall)`

where precision is the proportion of true positive predictions out of all positive predictions made by the model, and recall is the proportion of true positive predictions out of all actual positive samples in the dataset.

The F1 score ranges from 0 to 1, where a score of 1 indicates perfect precision and recall, and a score of 0 indicates no true positive predictions. A higher F1 score indicates better performance of the model.

The F1 score is useful when both precision and recall are important and a trade-off needs to be made between them. For example, in a medical diagnosis task, both high precision (few false positive predictions) and high recall (few false negative predictions) are important. The F1 score can be used to find a balance between precision and recall by adjusting the classification threshold of the model.

In summary, the F1 score is a commonly used performance metric in Machine Learning that combines precision and recall into a single score, and is useful when both precision and recall are important.

