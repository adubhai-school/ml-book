\section{What is explainable AI?}
Explainable AI (XAI) is a subfield of artificial intelligence (AI) that focuses on developing methods and techniques to make the decision-making process of AI systems transparent and understandable to humans. The goal of XAI is to create AI systems that can provide clear explanations for their actions and decisions, allowing users to understand and trust their behavior.

XAI is important because many AI systems, such as deep neural networks, operate as black boxes, meaning that it is difficult for humans to understand how they arrive at their decisions. This lack of transparency can be problematic in areas such as healthcare, finance, and criminal justice, where decisions made by AI systems can have significant impacts on people's lives.

One approach to XAI is to develop models that can provide interpretable representations of their internal workings. For example, decision trees and rule-based systems can provide clear and concise explanations for their decisions based on a set of rules. Another approach is to use techniques such as saliency maps and attention mechanisms to visualize the input features that are most relevant to the output of the model.

Another important aspect of XAI is the development of methods for evaluating the performance and reliability of AI systems. This includes techniques for detecting and diagnosing errors, verifying the correctness of the system, and assessing its robustness to different inputs and environments.

XAI has become increasingly important as AI is being deployed in more critical and complex applications, such as autonomous vehicles, medical diagnosis, and financial trading. By making AI systems more transparent and interpretable, XAI can help to build trust and confidence in AI and ensure that it is used in a responsible and ethical manner.

