\section{What is an SVM?}
SVM (Support Vector Machine) is a popular supervised machine learning algorithm used for classification and regression analysis. SVM works by finding a hyperplane in a high-dimensional space that separates the data points of different classes with the largest possible margin.

In a binary classification problem, SVM tries to find a hyperplane that separates the two classes of data points with the largest possible margin, which is the distance between the hyperplane and the closest data points of each class. SVM also allows for some misclassification errors by introducing a soft margin, which permits some data points to be on the wrong side of the hyperplane.

SVM can be used with different kernel functions, such as linear, polynomial, radial basis function (RBF), and sigmoid functions, to handle non-linearly separable data.

The SVM algorithm works as follows:

1. Convert the input data into a high-dimensional feature space.

2. Select a kernel function that defines the similarity between pairs of data points in the feature space.

3. Find the hyperplane that separates the data points with the largest possible margin.

4. Introduce a soft margin that allows for some misclassification errors.

5. Train the SVM model by finding the optimal hyperplane parameters that minimize the classification error.

6. Use the trained model to predict the class labels of new data points.

SVM is widely used in many applications, such as image classification, text classification, and bioinformatics. It has the advantage of being able to handle high-dimensional data and non-linear decision boundaries effectively. However, SVM can be sensitive to the choice of kernel function and parameters, and it can be computationally expensive for large datasets.

