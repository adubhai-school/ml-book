\section{What is a convolutional layer?}
A convolutional layer is a type of layer in a neural network that performs convolution operations on the input data. It is primarily used in deep learning applications for processing images, videos, and other multi-dimensional data that have spatial relationships.

In a convolutional layer, the input data is convolved with a set of learnable filters, also known as kernels or weights. Each filter extracts a specific feature or pattern from the input data by sliding or scanning over the input data and performing element-wise multiplication and summation operations.

The main advantages of using convolutional layers are:

1. Parameter sharing: The same set of filters can be applied to different parts of the input data, which reduces the number of parameters in the model and improves the generalization of the model.

2. Local connectivity: The filters only look at a small region of the input data at a time, which captures the local spatial relationships and enables the model to learn spatially invariant features.

3. Translation invariance: The output of the convolutional layer is invariant to translations of the input data, which makes the model more robust to small variations in the input data.

Convolutional layers can be stacked together to form a convolutional neural network (CNN), which is a type of neural network that is particularly well-suited for processing images and videos. CNNs have been shown to achieve state-of-the-art performance in many computer vision tasks, such as object recognition, segmentation, and detection.

