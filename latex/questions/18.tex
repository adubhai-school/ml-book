\section{What is dropout, and why is it used?}
Dropout is a regularization technique used in Deep Learning to prevent overfitting of the neural network. It involves randomly dropping out or deactivating a fraction of the neurons in a layer during training to reduce the co-adaptation of neurons and force the network to learn more robust and generalized features.

During each training iteration, dropout randomly sets a fraction of the neurons in a layer to zero, effectively removing them from the network for that iteration. The neurons that are dropped out are different for each iteration, making it difficult for the network to rely on any one particular set of neurons and forcing it to learn more generalized features.

Dropout can be applied to any type of layer in a neural network, including fully connected layers, convolutional layers, and recurrent layers. The dropout rate, which is the fraction of neurons that are dropped out at each iteration, is a hyperparameter that needs to be tuned based on the specific task and input data.

Dropout has been shown to be an effective regularization technique that can improve the performance and generalization of the network, especially for large and complex models. Dropout has also been shown to be useful in reducing the computational cost and memory requirements of the network by acting as a form of model compression.

In summary, Dropout is a regularization technique used in Deep Learning to prevent overfitting of the neural network by randomly dropping out a fraction of the neurons in a layer during training. Dropout can improve the performance and generalization of the network and reduce its computational cost and memory requirements.

