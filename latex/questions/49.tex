\section{What is a discriminative model?}
A discriminative model is a type of machine learning model that learns to predict the output variable based on the input variables. The goal of a discriminative model is to learn the decision boundary that separates the different classes or categories of the output variable.

Unlike generative models that model the underlying probability distribution of the data, discriminative models directly learn the mapping between the input variables and the output variable. This makes discriminative models more suitable for tasks such as classification, regression, and ranking, where the main objective is to predict the output variable based on the input variables.

Examples of discriminative models include logistic regression, support vector machines (SVMs), decision trees, and neural networks. These models are typically trained using supervised learning algorithms, where the model is provided with labeled training data and learns to predict the output variable based on the input variables.

Discriminative models have many applications in machine learning and artificial intelligence, such as:

\subsection{- Classification:}  Discriminative models can be used to classify data into different categories, such as spam vs. non-spam emails, or benign vs. malignant tumors.

\subsection{- Regression:}  Discriminative models can be used to predict a continuous output variable, such as the price of a house based on its features.

\subsection{- Ranking:}  Discriminative models can be used to rank items based on their relevance to a query, such as search results or recommendation systems.

\subsection{- Natural language processing:}  Discriminative models can be used to perform tasks such as named entity recognition, sentiment analysis, and machine translation.

Overall, discriminative models are a powerful tool in machine learning and can be used in a wide range of applications where predicting the output variable based on the input variables is important.

