\section{What is adversarial training?}
Adversarial training is a technique used in machine learning to improve the robustness and security of a model by training it to resist adversarial attacks. Adversarial attacks are intentional modifications made to input data that can cause a machine learning model to produce incorrect or unexpected results.

In adversarial training, the model is trained on both clean data and adversarial examples. Adversarial examples are generated by applying small perturbations to the input data, such as adding noise or making small changes to the pixel values in an image. The model is then trained to correctly classify both the clean data and the adversarial examples.

The goal of adversarial training is to make the model more resilient to attacks, by exposing it to a wide range of possible inputs and forcing it to learn to recognize and reject adversarial examples. This can help to improve the robustness and security of machine learning models, especially in applications where security and reliability are critical, such as autonomous vehicles and medical diagnosis.

Adversarial training has become an active area of research in machine learning, and has led to the development of new techniques for generating adversarial examples and training models to resist them. However, it is important to note that adversarial training is not a silver bullet, and there is still much research to be done to improve the robustness and security of machine learning models in the face of increasingly sophisticated attacks.

