\section{What is a fully connected layer?}
A fully connected layer is a type of layer in a neural network where all the neurons in one layer are connected to all the neurons in the next layer. It is also known as a dense layer or a linear layer.

In a fully connected layer, each neuron in the current layer receives an input from all the neurons in the previous layer and computes a weighted sum of the inputs, followed by a non-linear activation function. The weights of the connections between the neurons are learnable parameters that are adjusted during the training phase to minimize the loss function.

Fully connected layers are commonly used in the final layers of a neural network to map the output of the previous layers to the desired output of the network, such as a class label or a regression value. They can also be used in the intermediate layers of a neural network to learn complex and non-linear relationships between the input data and the output.

The main advantages of using fully connected layers are:

1. Flexibility: Fully connected layers can learn complex and non-linear functions of the input data, which makes them suitable for a wide range of machine learning tasks, such as classification, regression, and generation.

2. Robustness: Fully connected layers can capture high-level features and representations of the input data, which makes them more robust to variations and noise in the data.

3. Interpretable: Fully connected layers can provide insights into the relationships between the input data and the output by visualizing the learned weights and activations of the neurons.

Fully connected layers are widely used in many types of neural networks, such as feedforward neural networks, autoencoders, and generative adversarial networks (GANs).

