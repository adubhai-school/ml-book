\section{What is the EM algorithm?}
The EM algorithm (Expectation-Maximization algorithm) is a statistical algorithm used to estimate the parameters of a statistical model when some of the data is missing or incomplete. It is an iterative algorithm that alternates between two steps: the E-step (Expectation step) and the M-step (Maximization step).

In the E-step, the algorithm estimates the missing data by computing the expected value of the missing data given the observed data and the current parameter estimates. This is done by computing the posterior probability of the missing data using Bayes' rule and the current parameter estimates.

In the M-step, the algorithm updates the parameter estimates by maximizing the likelihood function based on the observed data and the estimated missing data. This is done using standard optimization techniques, such as gradient descent or Newton's method.

The EM algorithm iterates between the E-step and the M-step until convergence, that is, until the change in the parameter estimates is smaller than a predefined threshold.

The EM algorithm is widely used in various fields, such as signal processing, image analysis, natural language processing, and machine learning. It is particularly useful when dealing with missing or incomplete data, such as in data imputation, clustering, and latent variable modeling.

However, the EM algorithm has some limitations, such as the assumption of a specific statistical model, the sensitivity to the choice of the initial parameter estimates, and the possibility of converging to a local rather than global optimum. These limitations can be addressed by using more advanced variants of the EM algorithm, such as the stochastic EM algorithm or the variational EM algorithm.

