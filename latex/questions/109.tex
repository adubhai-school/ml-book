\section{What is a hyperparameter tuning package?}
A hyperparameter tuning package is a software library or tool that provides functionalities for tuning the hyperparameters of a machine learning model. These packages typically offer a range of hyperparameter optimization algorithms and strategies, as well as interfaces to different machine learning frameworks or libraries. They aim to simplify the process of hyperparameter tuning by automating the search for optimal hyperparameters and providing tools to monitor and visualize the results.

Examples of popular hyperparameter tuning packages include Hyperopt, Optuna, Ray Tune, and Scikit-Optimize. These packages offer a variety of optimization algorithms, such as Bayesian optimization, random search, and grid search, and can be used with popular machine learning frameworks such as TensorFlow, PyTorch, and Scikit-Learn. They typically provide a high-level interface that abstracts away the implementation details of the optimization algorithm and allows users to define a search space for the hyperparameters, set up constraints, and specify the optimization objective.

Hyperparameter tuning packages are particularly useful when dealing with complex models with a large number of hyperparameters or when training on large datasets. They can help to automate the search for optimal hyperparameters, reducing the need for manual tuning and allowing users to focus on other aspects of the machine learning pipeline.
