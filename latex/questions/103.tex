\section{What is a hyperparameter tuning strategy?}
A hyperparameter tuning strategy is a systematic approach to finding the best hyperparameters for a machine learning model. Hyperparameters are settings that are set prior to training and cannot be learned from data, such as the learning rate or the number of hidden layers in a neural network.

Hyperparameter tuning is an important step in building effective machine learning models, as the choice of hyperparameters can have a significant impact on the performance of the model.

There are several different strategies that can be used for hyperparameter tuning, including manual tuning, grid search, random search, and Bayesian optimization.

Manual tuning involves manually adjusting hyperparameters based on prior knowledge or trial and error. This approach is often time-consuming and may not be feasible for models with a large number of hyperparameters.

Grid search involves exhaustively searching a predefined set of hyperparameters to find the best combination. This approach can be effective for models with a small number of hyperparameters, but can be computationally expensive for models with a large number of hyperparameters.

Random search involves randomly sampling hyperparameters from a predefined distribution. This approach can be more efficient than grid search for models with a large number of hyperparameters, as it focuses on sampling regions of the hyperparameter space that are more likely to yield good results.

Bayesian optimization involves using probabilistic models to efficiently explore the hyperparameter space. This approach can be more efficient than random search, as it adapts to the results of previous evaluations to sample hyperparameters that are more likely to improve the model performance.

