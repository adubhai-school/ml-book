\section{What is a Gaussian mixture model?}
A Gaussian mixture model (GMM) is a statistical model used for density estimation and clustering of multivariate data. It assumes that the observed data are generated by a mixture of several Gaussian distributions with different means and variances.

The GMM consists of a weighted sum of K Gaussian distributions, each with its own mean vector and covariance matrix. The weights represent the relative proportions of each component in the mixture, and the covariance matrices represent the shape and orientation of the distribution for each component.

The GMM is typically estimated using the Expectation-Maximization (EM) algorithm, where the parameters of the model, including the weights, means, and covariance matrices, are iteratively updated to maximize the log-likelihood of the observed data.

The GMM can be used for a variety of tasks, such as density estimation, clustering, and outlier detection. It has many desirable properties, such as the ability to model complex distributions, the flexibility to handle data with varying shapes and sizes, and the ability to estimate the number of clusters automatically.

However, the GMM also has some limitations, such as the assumption of Gaussian distributions for each component, the sensitivity to the choice of the number of components and the initialization of the parameters, and the computational complexity of the EM algorithm. These limitations can be addressed by using more advanced variants of the GMM, such as the Bayesian GMM or the variational GMM.

