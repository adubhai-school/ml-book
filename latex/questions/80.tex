\section{What is federated learning?}
Federated learning is a machine learning approach that enables multiple parties to collaborate on the development of a shared model without sharing their data directly. The basic idea is to train a model on distributed data sources without transmitting the data to a central location. Instead, the model is sent to each data source, and the data sources use their local data to improve the model. The updated model is then sent back to the central location, where it is combined with the models from other data sources to create a new, improved model.

The advantages of federated learning include increased privacy, reduced communication costs, and the ability to learn from distributed data sources. By keeping the data local and not transmitting it, federated learning can help to address privacy concerns and regulatory issues, such as those related to medical and financial data. It can also reduce the amount of data that needs to be transmitted, which can be important in low-bandwidth environments.

Federated learning is often used in applications such as mobile devices and the Internet of Things (IoT), where data is generated and stored locally on devices with limited resources. By training models on distributed data sources, federated learning can help to improve the performance and accuracy of models while minimizing the amount of data that needs to be transmitted.

