\section{What is model compression?}
Model compression is a technique used in machine learning to reduce the size and computational complexity of a trained model while maintaining or improving its performance. The goal of model compression is to make machine learning models more efficient, so that they can be deployed on devices with limited resources, such as mobile phones and embedded systems.

There are several techniques for model compression, including:

1. Pruning: This involves removing the least important weights and neurons in a neural network. The remaining network can be smaller and faster to run, but still maintain its accuracy.

2. Quantization: This involves reducing the precision of weights and activations in a neural network, such as converting 32-bit floating-point numbers to 8-bit integers. This can reduce the size of the model and make it faster to run.

3. Knowledge distillation: This involves training a smaller model to mimic the behavior of a larger, more complex model. The smaller model can learn from the knowledge and insights of the larger model, while still being much smaller and faster to run.

4. Compact architectures: This involves designing smaller and more efficient models from the ground up, such as mobile-friendly versions of popular neural network architectures like VGG, ResNet, and Inception.

Model compression is becoming increasingly important as machine learning is deployed on devices with limited resources, such as smartphones, IoT devices, and edge computing platforms. By reducing the size and complexity of models, model compression can help to make machine learning more accessible and practical for a wide range of applications.

