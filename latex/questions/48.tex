\section{What is a generative model?}
A generative model is a type of machine learning model that learns to generate new data samples that are similar to the training data. The goal of a generative model is to model the underlying distribution of the training data and use that knowledge to generate new data samples that are indistinguishable from the original data.

There are two main types of generative models:

1. Explicit generative models: These models directly learn the probability distribution of the training data and use it to generate new samples. Examples of explicit generative models include Naive Bayes, Gaussian Mixture Models, and Autoregressive Models.

2. Implicit generative models: These models learn a function that can sample from the underlying distribution of the training data, without explicitly computing the probability distribution. Examples of implicit generative models include Generative Adversarial Networks (GANs) and Variational Autoencoders (VAEs).

Generative models have many applications in machine learning and artificial intelligence, such as:

1. Data generation: Generative models can be used to generate new data samples that are similar to the training data, such as images, videos, and text.

2. Data augmentation: Generative models can be used to augment the training data by generating new samples that can be used to train a model more effectively.

3. Anomaly detection: Generative models can be used to detect anomalies or outliers in the data by modeling the normal distribution of the data and identifying samples that deviate from it.

4. Data compression: Generative models can be used to compress the data by learning a low-dimensional representation of the data that can be used to reconstruct the original data.

Generative models are a powerful tool in machine learning and can be used in a wide range of applications where generating new data samples is important.

