\section{What is online learning?}
Online learning is a machine learning approach that involves updating a model continuously as new data becomes available, rather than training the model on a fixed dataset. The basic idea is to use incoming data to update the model incrementally, allowing the model to adapt and improve over time.

In online learning, the model is initially trained on a small subset of the available data, and is then updated as new data becomes available. The updated model is then used to make predictions on new data, and the process is repeated continuously. This allows the model to learn and adapt to changes in the data over time, without requiring the model to be retrained on the entire dataset.

The advantages of online learning include increased efficiency, reduced memory requirements, and the ability to handle data streams in real-time. By updating the model incrementally, online learning can help to reduce the amount of memory required to store the model, and can improve the efficiency of the learning process. It can also be used to handle data streams in real-time, allowing the model to adapt and improve as new data becomes available.

Online learning is often used in applications such as recommendation systems, fraud detection, and anomaly detection, where the data is constantly changing and the model needs to adapt to new patterns and trends. By continuously updating the model, online learning can help to improve the accuracy and performance of machine learning models in dynamic and rapidly changing environments.

