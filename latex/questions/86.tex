\section{What is a value function?}
In reinforcement learning, a value function is a function that estimates the expected cumulative reward that an agent can obtain by following a particular policy in a given state or set of states. The value function provides a measure of how good or bad it is to be in a particular state under a particular policy.

The value function can be defined in terms of the expected cumulative reward over time, starting from a given state and following a particular policy. There are two main types of value functions: state-value functions and action-value functions.

A state-value function estimates the expected cumulative reward starting from a particular state and following a particular policy. It provides a measure of how good or bad it is to be in that state under that policy.

An action-value function estimates the expected cumulative reward starting from a particular state, taking a particular action, and then following a particular policy. It provides a measure of how good or bad it is to take a particular action in that state under that policy.

Value functions can be estimated using dynamic programming algorithms such as value iteration and policy iteration, or by using model-free reinforcement learning algorithms such as Q-learning or SARSA. The value function is an important component of many reinforcement learning algorithms, and is used to guide the agent's decision-making process.

