\section{What is active learning?}
Active learning is a machine learning approach that involves selecting the most informative or relevant data samples for training a model. The basic idea is to choose the data samples that will help the model to learn the most, rather than simply using a random subset of the available data.

In active learning, the model is initially trained on a small subset of the available data. The model is then used to make predictions on the remaining data, and the samples for which the model is uncertain or most likely to make errors are selected for inclusion in the training set. The updated model is then trained on the new data, and the process is repeated until the model achieves a desired level of performance.

The advantages of active learning include increased efficiency, reduced data labeling costs, and improved accuracy. By selecting the most informative data samples, active learning can help to reduce the amount of data that needs to be labeled, which can be expensive and time-consuming. It can also help to improve the accuracy of the model, especially in cases where the data is sparse or imbalanced.

Active learning is often used in applications such as natural language processing, image recognition, and anomaly detection, where there is a large amount of unlabeled data and limited resources for labeling. By selecting the most informative data samples for labeling, active learning can help to improve the performance and efficiency of machine learning models.

