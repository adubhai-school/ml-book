\section{What is a policy?}
In reinforcement learning, a policy is a function that maps states to actions. It defines the behavior of the agent in an environment by specifying what action the agent should take in each state.

The goal of the agent is to learn a policy that maximizes the expected cumulative reward over time. This can be achieved using dynamic programming algorithms such as value iteration and policy iteration, or by using model-free reinforcement learning algorithms such as Q-learning or SARSA.

A policy can be deterministic or stochastic. A deterministic policy specifies a single action for each state, while a stochastic policy specifies a probability distribution over actions for each state.

The choice of policy can have a significant impact on the performance of the agent. A well-designed policy can help the agent to navigate complex environments and achieve high levels of performance, while a poorly designed policy can result in suboptimal behavior and low levels of performance.

Various techniques can be used to learn or optimize a policy in reinforcement learning, including value-based methods, policy gradient methods, and actor-critic methods. The choice of technique depends on the specifics of the problem and the characteristics of the environment.

