\section{What is a deep belief network?}
A deep belief network (DBN) is a type of neural network that consists of multiple layers of restricted Boltzmann machines (RBMs). It is a generative model that learns to represent the probability distribution of a set of input data in a hierarchical manner, where each layer learns increasingly abstract features of the data.

The DBN consists of an input layer, multiple hidden layers of RBMs, and an output layer. The RBMs in the hidden layers are trained using unsupervised learning to learn a compressed representation of the input data. The output layer is typically a supervised layer, such as a softmax layer for classification tasks.

The DBN is trained layer by layer using a greedy layer-wise unsupervised learning algorithm, where each layer is trained independently using Contrastive Divergence, and the weights are fine-tuned using supervised learning. The learning algorithm uses stochastic gradient descent to find the weights that maximize the log-likelihood of the observed data.

The DBN can be used for a variety of tasks, such as image recognition, speech recognition, and natural language processing. It has many desirable properties, such as the ability to learn high-level abstractions of the input data, the ability to handle missing or incomplete data, and the ability to generate new samples from the learned distribution.

However, the DBN also has some limitations, such as the difficulty of training the model for large datasets, the sensitivity to the choice of hyperparameters, and the need for a large amount of training data. These limitations can be addressed by using more advanced variants of the DBN, such as the Convolutional DBN or the Recurrent DBN.

