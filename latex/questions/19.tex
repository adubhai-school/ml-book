\section{What is ensemble learning?}
Ensemble learning is a technique in Machine Learning that involves combining multiple models to improve the accuracy and performance of the overall system. Ensemble learning can be used with various types of models, including decision trees, neural networks, and support vector machines.

The basic idea behind ensemble learning is that combining multiple models can reduce the risk of overfitting and increase the stability and robustness of the system. Ensemble learning can be used in two ways:

\subsection{- Bagging:}  In this approach, multiple models are trained independently on different subsets of the training data, and the outputs of the models are combined by averaging or voting. Bagging can help to reduce the variance of the models and improve the overall accuracy and generalization of the system.

\subsection{- Boosting:}  In this approach, multiple models are trained iteratively, where each subsequent model is trained on the misclassified samples of the previous model. The outputs of the models are combined using weighted averaging, where the weight of each model is proportional to its accuracy. Boosting can help to reduce the bias of the models and improve the overall accuracy and robustness of the system.

Ensemble learning can also be used with different types of models, such as heterogeneous ensembles, where multiple types of models are combined, and stacked ensembles, where multiple models are combined using another model as a meta-learner.

Ensemble learning has been shown to be a highly effective technique in many domains, such as image classification, object detection, and natural language processing. Ensemble methods have achieved state-of-the-art performance in many benchmarks, and are widely used in industry and research to improve the accuracy and reliability of Machine Learning systems.

