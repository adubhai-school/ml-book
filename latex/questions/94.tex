\section{What is deep reinforcement learning?}
Deep reinforcement learning is a subfield of machine learning that combines reinforcement learning with deep neural networks to learn policies for sequential decision-making problems. In deep reinforcement learning, the agent uses a deep neural network to estimate the value function or policy, which allows it to handle high-dimensional state and action spaces.

Deep reinforcement learning has been successfully applied to a wide range of complex tasks, such as game playing, robotics, and autonomous driving. Some of the most famous examples of deep reinforcement learning include AlphaGo and AlphaZero, which used deep reinforcement learning to master the game of Go and other games from scratch.

One of the key challenges of deep reinforcement learning is the trade-off between exploration and exploitation. Deep reinforcement learning algorithms can easily get stuck in local optima or learn suboptimal policies if they do not explore the environment sufficiently. To address this challenge, researchers have developed various techniques such as epsilon-greedy exploration, Monte Carlo tree search, and intrinsic motivation.

Another challenge of deep reinforcement learning is the difficulty of training deep neural networks with reinforcement learning. Deep reinforcement learning requires large amounts of data and computationally expensive algorithms, which can make training slow and unstable. To address this challenge, researchers have developed various techniques such as experience replay, target networks, and actor-critic methods.

