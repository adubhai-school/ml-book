\section{What is Monte Carlo simulation?}
Monte Carlo simulation is a computational technique that uses random sampling to model and analyze complex systems or processes. It is named after the famous Monte Carlo casino in Monaco, which is known for its games of chance.

In a Monte Carlo simulation, a large number of random samples are generated from a probability distribution that represents the uncertainty or variability in the system or process being modeled. These random samples are then used to estimate the behavior of the system or process, and to calculate the probabilities of different outcomes.

Monte Carlo simulation can be used to solve problems in a wide range of fields, including finance, engineering, physics, and biology. It is often used to model systems or processes that are too complex to analyze using traditional analytical methods.

Monte Carlo simulation can be used for a variety of purposes, such as estimating the value of a financial derivative, simulating the behavior of a physical system, or predicting the outcomes of a medical treatment. It can also be used to perform sensitivity analysis and identify the factors that have the greatest impact on the behavior of the system or process.

Monte Carlo simulation is a powerful tool for decision making under uncertainty, and can help decision makers to make informed decisions in complex and uncertain environments.

