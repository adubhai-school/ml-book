\section{What is a neural architecture search?}
Neural architecture search (NAS) is a type of automated machine learning (AutoML) technique that involves automatically discovering the optimal neural network architecture for a given task.

Traditionally, the architecture of a neural network is designed by humans, who use their knowledge and experience to determine the number of layers, the size of each layer, the activation functions, and other hyperparameters that are important for the network's performance. However, designing an optimal neural network architecture is a difficult and time-consuming process, especially for complex tasks.

Neural architecture search automates this process by using search algorithms to explore a large space of possible neural network architectures and selecting the one that performs the best on a given task. This can be done using various search methods, such as reinforcement learning, evolutionary algorithms, and gradient-based optimization.

The advantage of neural architecture search is that it can discover novel and effective network architectures that are not obvious to human designers. This can lead to improved performance on various tasks, such as image classification, natural language processing, and speech recognition. However, neural architecture search can be computationally expensive and requires large amounts of computational resources.

Despite its challenges, neural architecture search is a rapidly growing field of research and has the potential to automate the design of neural networks for a wide range of applications.

