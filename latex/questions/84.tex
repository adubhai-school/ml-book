\section{What is the Markov decision process?}
The Markov Decision Process (MDP) is a mathematical framework used to model decision-making processes in situations where outcomes are uncertain and dependent on previous decisions. It is commonly used in reinforcement learning to model the interactions between an agent and its environment.

An MDP is defined by a set of states, actions, and transition probabilities. At each time step, the agent is in a particular state and can take an action, which may result in a reward and a transition to a new state. The probability of transitioning to the new state depends on the current state and the action taken.

The goal of the agent is to learn a policy that maximizes the expected cumulative reward over time. This can be achieved using dynamic programming algorithms such as value iteration and policy iteration, or by using model-free reinforcement learning algorithms such as Q-learning or SARSA.

MDPs are widely used in applications such as robotics, autonomous vehicles, and game playing, where the agent must make decisions in a complex and uncertain environment. The MDP framework provides a powerful tool for modeling and solving these problems, and has been instrumental in advancing the field of reinforcement learning.

