\section{What is the role of activation functions in neural networks?}
Activation functions play a critical role in Neural Networks by introducing non-linearities and enabling the network to learn complex mappings between the input and output data. An activation function is applied to the output of each neuron in the network and determines whether the neuron should be activated or not based on its input.

The activation function takes the weighted sum of the input signals from the previous layer (or the input layer) and adds a bias term to it. It then applies a non-linear function to this sum, which introduces non-linearity into the network and allows it to learn complex functions.

Without activation functions, neural networks would be limited to linear transformations of the input data, which are not capable of modeling many real-world problems. With activation functions, the network can learn non-linear and complex functions, making it much more powerful and flexible.

There are several types of activation functions used in Neural Networks, including sigmoid, tanh, ReLU, Leaky ReLU, and Softmax. Sigmoid and tanh activation functions were popular in the past but have been largely replaced by ReLU and its variants due to their better performance in deep learning.

ReLU (Rectified Linear Unit) is currently the most popular activation function due to its simplicity and effectiveness. It has been shown to speed up the training process and improve the performance of deep neural networks significantly. Leaky ReLU is a variant of ReLU that solves the dying ReLU problem by introducing a small non-zero slope for negative input values.

In summary, activation functions are a critical component of Neural Networks that enable them to learn complex non-linear mappings between the input and output data. By introducing non-linearity into the network, activation functions allow it to model complex real-world problems and achieve high accuracy and performance.

