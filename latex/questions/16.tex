\section{What is a convolutional neural network?}
A Convolutional Neural Network (CNN) is a type of Artificial Neural Network (ANN) designed for image processing and computer vision tasks. CNNs are designed to automatically and adaptively learn spatial hierarchies of features from input images by using filters or kernels that are convolved with the input data to extract local features and patterns.

The architecture of a CNN consists of three types of layers:

\subsection{- Convolutional layer:}  This layer applies a set of filters or kernels to the input image to extract local features and patterns. The filters slide over the input image in a systematic way, and each filter produces a feature map that highlights a particular pattern in the input image.

\subsection{- Pooling layer:}  This layer reduces the spatial dimensionality of the feature maps produced by the convolutional layer by aggregating neighboring values. The most common type of pooling operation is max pooling, which selects the maximum value from each subregion of the feature map.

\subsection{- Fully connected layer:}  This layer takes the flattened output of the previous layers and processes it using a set of fully connected neurons, similar to a traditional feedforward neural network.

CNNs are trained using the backpropagation algorithm, where the network is first initialized with random weights and biases, and the backpropagation algorithm is used to iteratively adjust the weights and biases to minimize the difference between the predicted output and the true output.

CNNs have been very successful in many computer vision tasks, such as image classification, object detection, and segmentation. They can learn and extract local features and patterns in the input data and are highly effective at handling the high-dimensional and complex data structures in images. CNNs have also been used in other fields such as natural language processing and speech recognition by adapting their architecture to handle different types of inputs.

