\section{What is a data pipeline?}
A data pipeline is a sequence of steps or processes that extracts, transforms, and loads (ETL) data from one or more sources and prepares it for analysis or consumption by downstream applications. It is a key component of many data-driven systems and is particularly important in big data and machine learning applications.

A typical data pipeline consists of the following stages:

1. Data ingestion: This stage involves collecting and importing data from various sources, such as databases, files, APIs, and streaming platforms.

2. Data preprocessing: This stage involves cleaning, validating, and transforming the raw data into a format that is suitable for analysis or modeling. It may include tasks such as filtering, sorting, aggregating, and joining the data.

3. Feature extraction: This stage involves selecting and extracting relevant features from the preprocessed data that can be used as inputs to a machine learning model. It may include tasks such as dimensionality reduction, feature scaling, and feature engineering.

4. Model training: This stage involves training a machine learning model on the extracted features using a training set of labeled data. It may involve selecting an appropriate algorithm, tuning the hyperparameters, and evaluating the performance of the model.

5. Model deployment: This stage involves deploying the trained model into a production environment where it can be used to make predictions on new data.

6. Data visualization and reporting: This stage involves visualizing the results of the analysis or modeling and creating reports or dashboards that can be shared with stakeholders.

Data pipelines can be implemented using various tools and technologies, such as Apache Kafka, Apache Spark, Apache Airflow, and TensorFlow Data Validation. The design and implementation of a data pipeline depend on the specific requirements and constraints of the application, such as the volume, velocity, and variety of the data, the latency and throughput requirements, and the quality and reliability of the results.

