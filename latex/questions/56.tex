\section{What is a Boltzmann machine?}
A Boltzmann machine is a type of neural network that models the joint probability distribution of a set of binary variables. It is a type of energy-based model that uses the Boltzmann distribution from statistical mechanics to define the probability of each state of the network.

The Boltzmann machine consists of a set of visible nodes and hidden nodes, connected by weighted edges. The nodes can take on binary values (0 or 1), and the weights represent the strength of the interaction between the nodes.

The Boltzmann machine is trained using a learning algorithm called Contrastive Divergence, which involves updating the weights based on the difference between the observed data and the reconstructed data generated by the model. The learning algorithm uses stochastic gradient descent to find the weights that maximize the log-likelihood of the observed data.

The Boltzmann machine can be used for a variety of tasks, such as image recognition, speech recognition, and natural language processing. It has many desirable properties, such as the ability to model complex dependencies between variables, the ability to generate new samples from the learned distribution, and the ability to handle missing or incomplete data.

However, the Boltzmann machine also has some limitations, such as the difficulty of training the model for large datasets, the computational complexity of the learning algorithm, and the sensitivity to the choice of hyperparameters. These limitations can be addressed by using more advanced variants of the Boltzmann machine, such as the Deep Boltzmann Machine or the Restricted Boltzmann Machine.

